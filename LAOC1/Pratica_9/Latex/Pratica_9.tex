% Definição do tamanho da letra, folha e estilo.
\documentclass[12pt, a4paper]{article}

% Definição de pacotes.
%% Padrão UTF-8.
%% Texto brasileiro.
%% Geometria da página de acordo com a ABNT.
\usepackage[utf8]{inputenc}
\usepackage[brazil]{babel}
\usepackage{indentfirst}
\usepackage{geometry}
\geometry{a4paper, left = 3cm, right = 3cm, top = 3cm, bottom = 3cm}

% Numeração da página.
\pagenumbering{arabic}

\title{\textbf{\textit{Datapath} de um Processador Multi-ciclo}}
\author{
	Guimarães, João Guilherme M.\\
	\texttt{joaog95@live.com}
	\and
	Muniz, Lucas L. R.\\
	\texttt{lucaslc01@hotmail.com}
}
\date{\today}

\begin{document}
	% Escrever o título, autor e data.
	\maketitle
	
	% Espaçamento vertical
	\vspace{1cm}
	
	% Nova seção
	\section{Objetivo da prática}
	
	\par A nona aula prática da disciplina de Laboratório de Arquitetura de Computadores I, teve como objetivo realizar modificações no código do processador multi-ciclo desenvolvido na aula anterior, estas mudanças são:
	
	\begin{itemize}
		\item Os registradores no formato da instrução passam a ter 4 bits;

		\item Mudança na relação entre os registradores de cada instrução e

		\item Mudança no \textit{Opcode} de algumas instruções.
	\end{itemize}

    \section{Descrição das Atividades}

    \par O aumento de um bit de endereçamento no formato da instrução, permitiu que \textbf{RF} tivesse um acréscimo de 8 registradores, passando assim de 8 para 16, e para manter o padrão de projeto, o registrador referente ao \textbf{PC} também foi alterado, estando agora na posição 15, a última posição de \textbf{RF}.
    
    \vspace{\baselineskip}
    
    \par As alterações entre as relações dos registradores de cada instrução, acarretou na criação de uma nova variável de controle, \textit{WriteReg}, possibilitando com que as novas instruções possam salvar seus dados em um dos 3 registradores, \textbf{X}, \textbf{Y} e \textbf{Z}, sendo que antes, o registrador \textbf{X} era fixo.

    \vspace{\baselineskip}
    
    \par Um resumo do que foi descrito acima, se encontra listado logo abaixo.
    
    \begin{itemize}
		\item Expansão de \textbf{RF} de 8 para 16 registradores;

        \item Substituição do registrador referente ao \textbf{PC} para a posição 15;

        \item Criação da nova variável de controle \textit{WriteReg};
    \end{itemize}

    \section{Simulação}
    
    \par Após as modificações necessárias, iniciamos a simulação utilizando a instrução \textit{copy input}, devido ao fato da extrema necessidade de inicialização dos registradores nas demais instruções.
    
	\begin{itemize}
		\item[1 Copy Input] description
	\end{itemize}
	
	\section{Dificuldades}
	
	\par As principais dificuldades obtidas no desenvolvimento desta prática foram:
	
	\begin{enumerate}
		\item Encontrar e resolver o erro na instrução \textit{conditional copy}, já que a mesma realizava o inverso do proposto (executava o processo de \textit{copy} somente quando o valor na \textbf{ULA} era diferente de 0) e

        \item Realizar os testes de cada instrução após as alterações, o que resulta em um total de 9 testes.
	\end{enumerate}

	\section{Conclusão}
	
	\par Com a execução desta prática, pudemos aperfeiçoar nossos conhecimentos em projetos de processadores multi-ciclos e da linguagem Verilog, além de exercitar o trabalho em equipe.

\end{document}
	